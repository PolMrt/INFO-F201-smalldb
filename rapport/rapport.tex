\documentclass[utf8]{article}

\usepackage[utf8]{inputenc}

\usepackage[parfill]{parskip}

\usepackage{floatrow}
\usepackage{amsmath}
\usepackage{amssymb}
\usepackage{amsfonts}
\usepackage{graphicx}
\usepackage{float}
\usepackage{listingsutf8}
\usepackage{multirow}

\usepackage{fullpage}

% -----------------------------------------------------


\title{INFO-F201 SmallDB : rapport}
\author{MARKOWITCH Romain \and HENNEBICQ Jérémy\and MARNETTE Pol}
\date{16 décembre 2022}

\renewcommand{\contentsname}{Table des matières}
\begin{document}
\maketitle
\tableofcontents

\newpage

% -----------------------------------------------------

\section{Introduction}

Ce projet est la prolongation de la première partie "tinydb" en prenant une nouvelle approche. Le but est le même la réalisation d'une base de données en C++. Ici le serveur est différencié du client et communiquent ensemble via des socket. Aussi, les requêtes des différents clients seront maitenant gérés par des threads individuels.

\end{document}